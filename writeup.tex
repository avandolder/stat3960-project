\documentclass{article}
\title{STAT-3980 Project}
\author{Adam Vandolder}

\usepackage{multicol}

\begin{document}
    \pagenumbering{gobble}
    \maketitle

    \section{Objective}
    
    To simulate a number of trials of a stock investment strategy in order to
    determine whether or not it is an effective strategy in the long run.

    \section{Code}
    
    My Python script works by simulating the strategy 20 times, then outputting
    the results, along with the average profit. During the simulation, it
    iterates through 70 days, each time adding \$1000 worth of shares to a
    running total and calculating the share price for the next day.  The share
    price is calculated according to the given algorithm, using
    \texttt{random.random()}, which generates a uniform random variable on [0,
    1).  The simulation ultimately returns the share total times the current
    share price, subtracting the cost of buying the shares (in this case, 70 *
    1000).

    \section{Experiment results}

    Profit/Loss from 20 Trials
    \begin{multicols}{2}
    \begin{verbatim}
Trial #1:   $  14239.29
Trial #2:   $  11195.63
Trial #3:   $  30888.82
Trial #4:   $ -10446.46
Trial #5:   $  -3622.53
Trial #6:   $  21708.33
Trial #7:   $  -1422.30
Trial #8:   $   2528.35
Trial #9:   $   6165.58
Trial #10:  $  -1002.27
Trial #11:  $   1470.45
Trial #12:  $ -10493.18
Trial #13:  $   1382.47
Trial #14:  $  17257.54
Trial #15:  $  -5910.84
Trial #16:  $   7143.83
Trial #17:  $   1242.53
Trial #18:  $  -5447.22
Trial #19:  $ -21885.45
Trial #20:  $  -4588.64
    \end{verbatim}
    \end{multicols}
    \begin{verbatim}
Average Profit:   $   2520.20
    \end{verbatim}

    \section{Conclusions}
    
    In the long run, it appears that the strategy will make money.  This can be
    seen when running the simulation for a large number (say, $>$1000) of
    trials, as it will give a profit, usually of around \$2000.  This can also
    be determined by the fact that the strategy entails buying \$1000 of shares
    each day, meaning that you will buy less when the price is high and more
    when the price is low, so as long as the average share price you bought at
    is less than the final share price, you will make money. By examining the
    probabilities, it becomes clear that the price of an individual share will
    always be in-between \$4 and \$14, with the probability of the price
    increasing going from 100\% down to 0\% by 10\% at each dollar the price
    increases by. So by starting out with the share price at \$9, you are
    likely to make a profit in the long run. As such, investing in this way is
    a good strategy.

\end{document}
